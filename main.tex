\documentclass{acaces}

\usepackage{todonotes}

\begin{document}


\title{Data Aware\\Embedded Machine Learning
}

\author{
    Emil~Njor\addressnum{1}\extranum{1}
}

\address{1}{
    Danmarks Tekniske Universitet,
    Richard Petersens Plads,
    Bygning 322,
    Rum 111,
    2800 Kongens Lyngby,
    Denmark
}

\extra{1}{E-mail: emjn@dtu.dk}

\pagestyle{empty}


\begin{abstract}
    Abstract to be filled out
    
\end{abstract}

\keywords{Embedded Machine Learning, Data Aware Methods, Data Aware NAS, Data Aware Datasets, Data Aware Neural Network}

\section{Introduction}
Embedded Machine Learning, also known as tinyML, is a research field aiming to bring complex machine learning models to embedded devices. 
The field is still in its infancy, but has already shown promising results in a wide range of applications, such as keyword spotting, image classification, and predictive maintenance.

The main challenge of embedded machine learning is to make machine learning work in the constrained resources, such as memory, compute, and energy, of embedded devices.
Most research in the field has focused on reducing the resource consumption of the machine learning models themselves - especially resource hungry models such as neural networks.

This reduction of resource consumption has been achieved using methods such as quantization, pruning and neural architecture search\cite{njor2022primer}.

However, the resource consumption of the machine learning model is only a part of the resource consumption of the entire machine learning system.
A typical embedded machine learning tasks consists of collecting data from a sensor, pre-processing this data, running the machine learning model, and taking some action based on the output of the machine learning model.

In my research I work with the data granularity of the input data captured, pre-processed and given to the machine learning model.
At the core of this research is the idea that input data can be configured to be more or less resource intensive.

E.g., sound data can be captured at different sample rates.
Likewise, image data can be captured at different resolutions.
Higher sample rates and image resolutions will be more resource intensive, but will also contain more information that may be valuable to the machine learning model.
On the other hand, lower sample rates and image resolutions will be less resource intensive, but will also contain less information.

This idea can be used to reduce the resource consumption of an embedded machine learning system, but also to improve the performance of a machine learning system under some given resource constraints.
We have thought the name "Data Aware Embedded Machine Learning" suitable to describe this research direction.

In the following sections of this poster abstract I will describe my work and ideas for work in this research direction in more detail.
I will introduce the ideas of:
\begin{description}
    \item[Data Aware Neural Architecture Search] An extension of Neural Architecture Search that simultaneously searches for an optimal combination of a Neural Network Architecture and a Data Granularity.
    \item[Data Aware Datasets] Datasets that support easy loading of multiple data granularities.
    \item[Data Aware Neural Networks] Specialized Neural Networks that can run inference of multiple data granularities.
\end{description}

\section{Data Aware Neural Architecture Search}
The idea of a Data Aware Neural Architecture Search builds on top of a Neural Architecture Search.
Simply put, a Neural Architecture Search is a computer program that searches for an optimal Neural Network Architecture for a given task.

A Data Aware Neural Architecture Search extends this by searching for an optimal combination of a Data Granularity and Neural Network Architecture for a given task.
This extension enables scaling the resource consumption of the Data Granularity up or down to either give more information to the Neural Network Architecture, or to allocate resources to the Neural Network Architecture.
This should enable a Data Aware Neural Architecture Search to find Machine Learning Systems that perform better at a given resource consumption.

The idea of Data Granularity is not new.
Even so, many publications and works in the field of machine learning does not consider the Data Granularity of the input data. \todo[inline]{Should I cite something here?}
This can often lead to using more resources on data than necessary. 
E.g., a 2016 paper claims that sample rates used in the literature is up to 57\% higher than necessary \cite{khan2016optimising}.
Considering this, it is also likely that a Data Aware Neural Architecture Search can lower resource consumption for machine learning systems without sacrificing performance.

One disadvantage of a Data Aware Neural Architecture Search is that the search space is larger than a Neural Architecture Search.
Normally this would be a problem, as Neural Architectures are notoriously slow due to long training times of many models.
However, the Neural Network models used for embedded devices are often much smaller and faster to train which mitigates this issue.

I explored the idea of a Data Aware Neural Architecture Search in a 2023 paper of the same name \cite{njor2023data}.

\section{Data Aware Datasets}

% Many datasets can relatively simply be loaded at different data granularities
% Ease of use is important for adoption

\section{Data Aware Neural Networks}
% Per sample data granularity

\bibliography{bibliography}

\end{document}

