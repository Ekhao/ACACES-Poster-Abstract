\documentclass{acaces}

\begin{document}


\title{Data Aware\\Embedded Machine Learning
}

\author{
    Emil~Njor\addressnum{1}\extranum{1}
}

\address{1}{
    Danmarks Tekniske Universitet,
    Richard Petersens Plads,
    Bygning 322,
    Rum 111,
    2800 Kongens Lyngby,
    Denmark
}

\extra{1}{E-mail: emjn@dtu.dk}

\pagestyle{empty}


\begin{abstract}
    Abstract to be filled out
    % Embedded Machine Learning (also known as tinyML)
    % Data Aware Methods
    % Data Aware NAS
    % Data Aware Datasets
    % Data Aware Neural Network
    
\end{abstract}

\keywords{Embedded Machine Learning, Data Aware Methods, Data Aware NAS, Data Aware Datasets, Data Aware Neural Network}

\section{Introduction}
Embedded Machine Learning, also known as tinyML, is a research field aiming to bring complex machine learning models to embedded devices. 
The field is still in its infancy, but has already shown promising results in a wide range of applications, such as keyword spotting, image classification, and predictive maintenance.

The main challenge of embedded machine learning is to make machine learning work in the constrained resources of embedded devices, such as memory, compute, and energy.
Many traditional machine learning models such as support vector machines and decision trees can readily be implemented on embedded devices.
More complex models such as neural networks are more challenging to make fit the resource constraints.

The main approaches to making neural networks fit the resource constraints are as quantization and pruning.
% Maybe also mention Neural Architecture Search

% My Research
\section{Typesetter}

Use {\tt pdflatex guide.tex} to generate a PDF file, don't use {\tt latex} to
produce a dvi file and then {\tt dvipdf} or {\tt dvips} followed by {\tt
        ps2pdf} or {\tt pdftopdf} as this results in ugly looking PDF.

\section{Figures}
Your figures should also be made in PDF. Unfortunately, not a
lot of drawing applications allow you to write a PDF file. You can however
create EPS files and transform the resulting files to PDF using {\tt epstopdf
        figure.eps}. Figure~\ref{logo} shows an example of a figure.
Please bear in mind that your paper will be reduced to 70\% of its
original size.


\begin{figure}
    \centering
    \includegraphics{hipeac-logo-bw}
    \caption{An example of a Figure: the HiPEAC logo.}
    \label{logo}
\end{figure}


\section{Fonts}
Please make sure that your PDF file only contains Type1 fonts. This should be
no problem if you use pdflatex and the {\tt acaces.cls} style file. If you use
another typesetting system, please check if the resulting PDF file only uses
Type1 fonts.  You can check this using Acrobat Reader: open your file and
check all fonts using File$\rightarrow$Document Properties$\rightarrow$Fonts\ldots

\section{Final submission}
Make sure that your PDF file does not contain page numbers, is
up to 4 pages long and is formatted for an A4 page.
Upload the final version before June 8 on the website.

\section{About this style}
You are encouraged to send bug reports, remarks, \ldots about this style to
\href{mailto:ronsse@elis.UGent.be}{ronsse@elis.UGent.be}.

\section{End}
This is the end! \label{end}

\bibliography{bibliography}

\end{document}

